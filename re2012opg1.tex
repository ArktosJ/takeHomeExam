\documentclass{article}
\usepackage[utf8]{inputenc}
\usepackage{amssymb}
\usepackage[T1]{fontenc}
\usepackage[danish]{babel}
\usepackage{graphicx}
\usepackage{wrapfig}
\usepackage{listings}
\usepackage{xcolor}

\definecolor{codegreen}{rgb}{0,0.6,0}
\definecolor{codepurple}{rgb}{0.58,0,0.82}
\definecolor{codegray}{rgb}{0.5,0.5,0.5}
\definecolor{backcolour}{rgb}{0.99,0.99,0.97}

\lstdefinestyle{mystyle}{
    backgroundcolor=\color{backcolour},   
    commentstyle=\color{codegreen},
    keywordstyle=\color{magenta},
    numberstyle=\tiny\color{codegray},
    stringstyle=\color{codepurple},
    basicstyle=\ttfamily\footnotesize,
    breakatwhitespace=false,         
    breaklines=true,                 
    captionpos=b,                    
    keepspaces=true,                 
    numbers=left,                    
    numbersep=5pt,                  
    showspaces=false,                
    showstringspaces=false,
    showtabs=false,                  
    tabsize=2               
}

\lstset{style=mystyle}

\title{Reeksamen Januar 2012 opgave 1}
\author{julso21 }
\date{November 2021}

\begin{document}

\maketitle

\section{Opgave 1}
I denne opgave bliver to funktioner defineret: f: & $\mathbb{R}$ \rightarrow  & $\mathbb{R}$ og g: & $\mathbb{R}$ \rightarrow & $\mathbb{R}$ som er defineret således:
\\
   \(f(x) = x^2 + x + 1\) \\
   \(g(x) = 2x - 2\)
\\
\\
Del a lyder: Er f en bijektion?
\\
\\
Svar: Nej, det er f ikke i kraft af f's defination som en andengradsfunktion. Det er fordi at en negativ x-værdi vil stadig give en positiv y-værdi. F.eks. vil f(1) og f(-) begge give en y-værdi af 4, og et eneste punkt hvor dette er gælende betyder at f ikke er en bijektion   
\\
\\
Del b lyder: Har f en invers funktion?
\\
\\
Svar: Ja den ser således ud:
\\
\(y = x^2  + x + 1 \)\\
bliver til \\
\(y = {-1\pm\sqrt{4x-3}\over2} \)
\\Detter er dog ikke en gyldig funktion
\\
\\
Del c lyder: Angiv f + g
\\
\\
Svar: \(x^2+3x-1 \)
\\
\\
Del d lyder: Angiv g \circ f
\\
\\
Svar: \(2(x^2+x+1)-2 \)
\end{document}
